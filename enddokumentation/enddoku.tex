\documentclass[10pt]{scrartcl}
\author{Dane Wicki}
\title{Universal data acquisition}
\subtitle{FS17 Praxismodul}
\renewcommand{\contentsname}{Inhaltsverzeichnis}
\begin{document}
\maketitle
\tableofcontents
\newpage

\section{Einleitung, Problembeschreibung}
\subsection{Gescheftsfeld der Firma}
Die Siemens AG (Berlin und München) ist ein führender internationaler Technologiekonzern, der seit mehr als 165 Jahren für technische Leistungsfähigkeit, Innovation, Qualität, Zuverlässigkeit und Internationalität steht. Das Unternehmen ist in mehr als 200 Ländern aktiv, und zwar schwerpunktmäßig auf den Gebieten Elektrifizierung, Automatisierung und Digitalisierung. Siemens ist weltweit einer der größten Hersteller energieeffizienter ressourcenschonender Technologien. Das Unternehmen ist einer der führenden Anbieter effizienter Energieerzeugungs- und Energieübertragungslösungen, Pionier bei Infrastrukturlösungen sowie bei Automatisierungs-, Antriebs- und Softwarelösungen für die Industrie. Darüber hinaus ist das Unternehmen ein führender Anbieter bildgebender medizinischer Geräte wie Computertomographen und Magnetresonanztomographen sowie in der Labordiagnostik und klinischer IT.
\subsection{Projektkontext}
Die Firma Siemens BT in Zug ist zuständig für die Entwicklung von Brandmeldern.
Um die Qualität der Brandmelder zu gewährleisten, werden diese unter Zuhilfenahme
verschiedener Apparaturen und Testaufbauten getestet. Dies geschieht bei vielen Aufbauten automatisch und mit konsistenter Aufzeichnung der Daten, welche der Melder und etwaillige refferenzgeräte erzeugen. Es gibt jedoch weiterhin aufbauten, bei welchen die Aufzeichnung weder Automatisch noch Konsistent gespeichert werden kann oder nur unter grossen Anstrengungen der Arbeitenden. Diesen Zustand gilt es nun zu verbessern.
Dazu soll eine Software entwickelt werden, die aus verschiedenen Ressourcen (verschiedenen
Datenquellen) die Daten sammelt und diese in eine auswertbare Excel-Datei exportiert. Dies
Software basiert auf einer bestehenden Software, welche für das Brandlabor entwickelt wurde.
Es sollen dabei Bestandteile der dieser Bestehenden Software verwendet werden.
Es sind also folgende Ziele zu erfüllen:
\subsection{Problembeschreibung}
\subsection{Projektziele}
Siemens Building Technologies ist unter anderem ein Produkthersteller im Bereich Brandschutz. Dazu gehört die Herstellung verschiedener Brandmelder. Um sich der Qualität jener Brandmelder gewiss sein zu können, werden diese in verschiedenen eigens erstellten Testanlagen getestet. Sie werden getestet, um auch sicherzustellen das die Brandmelder bei der jeweiligen Zertifizierungstelle durchkommen. Bei einem Solchen Testdurchlauf werden alle Daten der jeweiligen Brandmelder aufgezeichnet und anschliessend ausgewertet.\newline
Es gibt eine grosse Variantion dieser Testanlagen. Bei den grösseren Testaufbauten wurden eigene Software erstellt, welche auch zur Automation genuzt werden.Es gibt jedoch auch einige etwas kleinere Testaufbauten, bei welcher keine Software vorhanden ist, welche die zu sammelnden Daten aufzeichnet. Es wird momentan bei jeder dieser Anlagen auf eine umständliche Art und Weise getestet.
Dies stellt eines der Probleme dar. Dieser umstand führt auf einen erhöten Zeitaufwand. Zudem kommt bei manchen Aufbauten dazu, dass sie selten gebraucht werden. Der fehlende Zyklische gebrauch der jener Aufbauten führt zu einer erhöten einarbeitungsperiode.
Weiter kommt ein neuer Standart für einen bestehenden Brandmelder hinzu. Dieser neue Standart führte dazu, dass die Testabteilung der Siemens einen neuen Testaufbau bei der Firma XYZ bestellte. Im Rahmen der bestellung wurde jedoch nur der Aufbau bestellt, keine Passende Software, welche alle Daten während eines Testlaufes aufzeichnen könnte. Für diesen Zweck gilt es eine Software zu entwickeln. 
\section{Projektergebnisse}
\subsection{Ergebnisse}
\subsection{Anforderungen}	

\section{Umsetzung}
\subsection{Bestehende Software}
\subsection{Bestehende Datenbank}
\subsection{Software Architektur}
\subsubsection{Recorder}
\subsubsection{GUI / Tasks}
\subsubsection{Objek}
\subsection{Datenbank}
\subsubsection{Datenbank-Classen}
\subsection{Testing}
\section{Projektergebnisse}
\section{Arbeitsjournal}
\section{Fazit}
\section{Bestätigung Arbeitgeber}

\end{document}